% ============================================
% INFORME DEL PROYECTO FINAL DE ESTADÍSTICA
% Universidad de La Habana - MATCOM
% Ciencia de la Computación - Curso 2025-2026
% ============================================

\documentclass[12pt,a4paper]{article}

% ============================================
% PAQUETES
% ============================================
\usepackage[utf8]{inputenc}
\usepackage[spanish,es-noshorthands]{babel}
\usepackage{amsmath,amssymb,amsthm}
\usepackage{graphicx}
\usepackage{booktabs}
\usepackage{hyperref}
\usepackage{geometry}
\usepackage{float}
\usepackage{caption}
\usepackage{subcaption}
\usepackage{xcolor}
\usepackage{listings}
\usepackage{tikz}
\usetikzlibrary{shapes,arrows,positioning,calc,fit,backgrounds}
\usepackage{pgfplots}
\pgfplotsset{compat=1.18}

\usepackage{enumitem}
\usepackage{tcolorbox}
\usepackage{multirow}
\usepackage{longtable}
\usepackage{array}
\usepackage{fancyhdr}

\geometry{margin=2.5cm}
\setlength{\headheight}{14.5pt}

% Configuración de encabezados
\pagestyle{fancy}
\fancyhf{}
\rhead{Proyecto Final de Estadística}
\lhead{Universidad de La Habana - MATCOM}
\rfoot{Página \thepage}

% Configuración de listings para Python
\lstset{
    language=Python,
    basicstyle=\ttfamily\small,
    keywordstyle=\color{blue},
    stringstyle=\color{red},
    commentstyle=\color{green!60!black},
    frame=single,
    breaklines=true,
    showstringspaces=false,
    numbers=left,
    numberstyle=\tiny\color{gray},
    backgroundcolor=\color{gray!5}
}

% Cajas para definiciones y conceptos importantes
\newtcolorbox{definicion}[1][]{
    colback=blue!5!white,
    colframe=blue!75!black,
    fonttitle=\bfseries,
    title=#1
}

\newtcolorbox{importante}[1][]{
    colback=yellow!10!white,
    colframe=orange!75!black,
    fonttitle=\bfseries,
    title=#1
}

\newtcolorbox{concepto}[1][]{
    colback=green!5!white,
    colframe=green!50!black,
    fonttitle=\bfseries,
    title=#1
}

% Definiciones de teoremas
\theoremstyle{definition}
\newtheorem{defn}{Definición}[section]
\newtheorem{ejemplo}{Ejemplo}[section]

% ============================================
% INFORMACIÓN DEL DOCUMENTO
% ============================================
\title{
    \vspace{-2cm}
    {\Large Universidad de La Habana}\\
    {\large Facultad de Matemática y Computación (MATCOM)}\\
    {\large Departamento de Matemática Aplicada}\\[1cm]
    \rule{\linewidth}{0.5mm}\\[0.4cm]
    {\LARGE \textbf{Análisis Estadístico del Mercado\\de Valores Tecnológico}}\\[0.3cm]
    {\large Estudio del Modelo CAPM y Comportamiento\\de Acciones del NASDAQ-100}\\[0.2cm]
    \rule{\linewidth}{0.5mm}\\[1.5cm]
    {\large \textbf{Proyecto Final de Estadística}}\\
    {\large Licenciatura en Ciencia de la Computación}\\
    {\large Curso 2025-2026}
}

\author{}
\date{\today}

\begin{document}

% Portada sin número de página
\maketitle
\thispagestyle{empty}
\newpage

% ============================================
% ÍNDICE
% ============================================
\tableofcontents
\newpage

% ============================================
% RESUMEN EJECUTIVO
% ============================================
\section*{Resumen Ejecutivo}
\addcontentsline{toc}{section}{Resumen Ejecutivo}

Este proyecto realiza un análisis estadístico integral de los rendimientos de 30 empresas tecnológicas líderes a nivel mundial durante el período 2018-2024. Utilizando datos históricos extraídos de Yahoo Finance, aplicamos diversas técnicas estadísticas para responder preguntas fundamentales sobre el comportamiento del mercado tecnológico.

\textbf{Principales hallazgos:}
\begin{itemize}
    \item Se estimaron los coeficientes beta (sensibilidad al mercado) para cada empresa mediante el modelo CAPM.
    \item Se identificaron grupos naturales de empresas con perfiles de riesgo-rendimiento similares usando K-Means.
    \item Se realizaron pruebas de hipótesis para comparar grupos de alto y bajo riesgo sistemático.
    \item Se aplicaron técnicas de bootstrap para obtener intervalos de confianza robustos.
\end{itemize}

\textbf{Técnicas estadísticas aplicadas:}
\begin{enumerate}
    \item Análisis Exploratorio de Datos (EDA)
    \item Regresión Lineal (Modelo CAPM)
    \item Análisis de Clustering (K-Means)
    \item Inferencia Estadística (t-tests, bootstrap, intervalos de confianza)
\end{enumerate}

\newpage

% ============================================
% CAPÍTULO 1: INTRODUCCIÓN
% ============================================
\section{Introducción}

\subsection{Motivación del Proyecto}

El mercado de valores representa uno de los sistemas más complejos de la economía moderna. Las acciones tecnológicas, en particular, han experimentado un crecimiento exponencial en las últimas décadas, convirtiéndose en el motor principal de los mercados financieros globales. Comprender el comportamiento de estas acciones y su relación con el mercado general es fundamental tanto para inversores individuales como para gestores de carteras profesionales.

Este proyecto aplica técnicas estadísticas avanzadas para analizar el comportamiento de las acciones que componen el sector tecnológico, utilizando como referencia el índice \textbf{NASDAQ-100} representado por el ETF \textbf{QQQ}.

\subsection{Preguntas de Investigación}

El presente estudio busca responder las siguientes preguntas mediante análisis estadístico riguroso:

\begin{importante}[Preguntas de Investigación]
\begin{enumerate}[label=\textbf{P\arabic*:}]
    \item \textbf{¿Cómo se relaciona el rendimiento de las acciones individuales con el rendimiento del mercado?}\\
    Se aborda mediante el modelo CAPM, cuantificando la sensibilidad (beta) de cada acción.
    
    \item \textbf{¿Existen grupos naturales de acciones con comportamientos similares?}\\
    Se utilizan técnicas de clustering para identificar patrones de agrupación.
    
    \item \textbf{¿Existen diferencias estadísticamente significativas entre acciones de alto y bajo riesgo sistemático?}\\
    Se aplican pruebas de hipótesis y métodos de bootstrap.
\end{enumerate}
\end{importante}

\subsection{Relevancia del Tema}

\begin{itemize}
    \item \textbf{Económica}: El NASDAQ-100 representa aproximadamente \$20 billones en capitalización de mercado.
    \item \textbf{Académica}: El modelo CAPM fue desarrollado por laureados Nobel como William Sharpe.
    \item \textbf{Práctica}: Los inversores utilizan estas métricas diariamente para decisiones de inversión.
    \item \textbf{Estadística}: Aplicación perfecta de regresión, inferencia y técnicas multivariantes.
\end{itemize}

\subsection{Estructura del Proyecto}

El proyecto está organizado en los siguientes notebooks de Python:

\begin{table}[H]
\centering
\begin{tabular}{lp{10cm}}
\toprule
\textbf{Archivo} & \textbf{Descripción} \\
\midrule
\texttt{extract\_yfinance\_data.py} & Script de extracción de datos de Yahoo Finance \\
\texttt{eda.ipynb} & Análisis Exploratorio de Datos \\
\texttt{regression.ipynb} & Modelo CAPM y regresión lineal \\
\texttt{clustering.ipynb} & Agrupación de empresas con K-Means \\
\texttt{inference.ipynb} & Inferencia estadística y pruebas de hipótesis \\
\bottomrule
\end{tabular}
\caption{Estructura de archivos del proyecto}
\end{table}

% ============================================
% CAPÍTULO 2: MARCO TEÓRICO
% ============================================
\section{Marco Teórico}

\subsection{Fundamentos de Finanzas Cuantitativas}

\subsubsection{Rendimientos Financieros}

El \textbf{rendimiento} de un activo financiero mide el cambio porcentual en su precio durante un período determinado. Existen dos formas principales de calcular rendimientos:

\begin{definicion}[Rendimiento Simple]
El rendimiento simple entre los períodos $t-1$ y $t$ se define como:
\begin{equation}
    R_t = \frac{P_t - P_{t-1}}{P_{t-1}} = \frac{P_t}{P_{t-1}} - 1
\end{equation}
donde $P_t$ es el precio del activo en el momento $t$.
\end{definicion}

\begin{definicion}[Rendimiento Logarítmico]
El rendimiento logarítmico (o rendimiento continuamente compuesto) se define como:
\begin{equation}
    r_t = \ln\left(\frac{P_t}{P_{t-1}}\right) = \ln(P_t) - \ln(P_{t-1})
\end{equation}
\end{definicion}

\begin{concepto}[¿Por qué usar rendimientos logarítmicos?]
Los rendimientos logarítmicos tienen propiedades estadísticas deseables:
\begin{enumerate}
    \item \textbf{Aditividad temporal}: $r_{t:t+n} = r_t + r_{t+1} + \cdots + r_{t+n}$
    \item \textbf{Simetría}: Un rendimiento de $+10\%$ seguido de $-10\%$ retorna al valor original.
    \item \textbf{Normalidad aproximada}: Facilitan el análisis estadístico paramétrico.
\end{enumerate}
\end{concepto}

\subsubsection{Volatilidad y Riesgo}

\begin{definicion}[Volatilidad]
La \textbf{volatilidad} ($\sigma$) es la desviación estándar de los rendimientos y mide la dispersión o incertidumbre:
\begin{equation}
    \sigma = \sqrt{\frac{1}{n-1}\sum_{t=1}^{n}(r_t - \bar{r})^2}
\end{equation}
donde $\bar{r}$ es el rendimiento promedio.
\end{definicion}

Para convertir volatilidad diaria a anual:
\begin{equation}
    \sigma_{\text{anual}} = \sigma_{\text{diario}} \times \sqrt{252}
\end{equation}
donde 252 es el número aproximado de días de trading en un año.

\subsection{El Modelo CAPM}

\subsubsection{Origen y Fundamento Teórico}

El \textbf{Capital Asset Pricing Model (CAPM)} fue desarrollado independientemente por:
\begin{itemize}
    \item William Sharpe (1964) - Premio Nobel de Economía 1990
    \item John Lintner (1965)
    \item Jan Mossin (1966)
\end{itemize}

El modelo se basa en el trabajo pionero de Harry Markowitz sobre teoría de portafolios.

\begin{definicion}[Modelo CAPM]
El rendimiento esperado de un activo $i$ está dado por:
\begin{equation}
    E[R_i] = R_f + \beta_i \cdot (E[R_m] - R_f)
\end{equation}
donde:
\begin{itemize}
    \item $E[R_i]$: Rendimiento esperado del activo $i$
    \item $R_f$: Tasa libre de riesgo
    \item $\beta_i$: Coeficiente beta del activo $i$
    \item $E[R_m]$: Rendimiento esperado del mercado
    \item $(E[R_m] - R_f)$: Prima de riesgo del mercado
\end{itemize}
\end{definicion}

\subsubsection{El Coeficiente Beta ($\beta$)}

El coeficiente beta es la medida central del CAPM y cuantifica el \textbf{riesgo sistemático} de un activo.

\begin{definicion}[Coeficiente Beta]
El beta de un activo $i$ respecto al mercado se define como:
\begin{equation}
    \beta_i = \frac{\text{Cov}(R_i, R_m)}{\text{Var}(R_m)} = \frac{\sigma_{i,m}}{\sigma_m^2}
\end{equation}
\end{definicion}

\begin{table}[H]
\centering
\begin{tabular}{cl}
\toprule
\textbf{Valor de $\beta$} & \textbf{Interpretación} \\
\midrule
$\beta = 1$ & El activo se mueve igual que el mercado \\
$\beta > 1$ & El activo es más volátil que el mercado (agresivo) \\
$\beta < 1$ & El activo es menos volátil que el mercado (defensivo) \\
$\beta < 0$ & El activo se mueve en dirección opuesta al mercado \\
$\beta = 0$ & El activo no tiene correlación con el mercado \\
\bottomrule
\end{tabular}
\caption{Interpretación de los valores del coeficiente beta}
\label{tab:beta_interpretacion}
\end{table}

\subsubsection{Estimación por Regresión Lineal}

En la práctica, el CAPM se estima mediante regresión lineal:
\begin{equation}
    R_{i,t} - R_f = \alpha_i + \beta_i (R_{m,t} - R_f) + \epsilon_{i,t}
\end{equation}

donde:
\begin{itemize}
    \item $\alpha_i$ (alfa): Rendimiento anormal (si $\alpha > 0$, el activo ``supera'' al mercado).
    \item $\beta_i$: Pendiente de la regresión (sensibilidad al mercado).
    \item $\epsilon_{i,t}$: Término de error (riesgo idiosincrático).
\end{itemize}

% Diagrama del CAPM
\begin{figure}[H]
\centering
\begin{tikzpicture}[scale=1.2]
    \begin{axis}[
        xlabel={Rendimiento del Mercado ($R_m - R_f$)},
        ylabel={Rendimiento del Activo ($R_i - R_f$)},
        xmin=-0.15, xmax=0.15,
        ymin=-0.2, ymax=0.25,
        grid=major,
        width=12cm,
        height=8cm,
        legend pos=north west
    ]
    
    % Línea de regresión (beta = 1.5)
    \addplot[thick, blue, domain=-0.12:0.12] {0.01 + 1.5*x};
    \addlegendentry{$\beta = 1.5$ (Agresivo)}
    
    % Línea de regresión (beta = 1.0)
    \addplot[thick, green!60!black, domain=-0.12:0.12] {0.005 + 1.0*x};
    \addlegendentry{$\beta = 1.0$ (Mercado)}
    
    % Línea de regresión (beta = 0.6)
    \addplot[thick, orange, domain=-0.12:0.12] {0.003 + 0.6*x};
    \addlegendentry{$\beta = 0.6$ (Defensivo)}
    
    \end{axis}
\end{tikzpicture}
\caption{Representación gráfica del CAPM para diferentes valores de beta}
\label{fig:capm_lines}
\end{figure}

\subsection{Análisis de Clustering}

\subsubsection{Algoritmo K-Means}

El algoritmo K-Means es una técnica de aprendizaje no supervisado para particionar $n$ observaciones en $k$ grupos.

\begin{definicion}[Algoritmo K-Means]
Dado un conjunto de observaciones $(\mathbf{x}_1, \mathbf{x}_2, \ldots, \mathbf{x}_n)$, K-Means busca minimizar la suma de cuadrados intra-cluster:
\begin{equation}
    \underset{S}{\arg\min} \sum_{i=1}^{k} \sum_{\mathbf{x} \in S_i} \|\mathbf{x} - \boldsymbol{\mu}_i\|^2
\end{equation}
donde $\boldsymbol{\mu}_i$ es el centroide del cluster $S_i$.
\end{definicion}

\subsubsection{Métricas de Evaluación de Clusters}

\begin{enumerate}
    \item \textbf{Silhouette Score}: Mide cohesión intra-cluster y separación inter-cluster.
    \begin{equation}
        s(i) = \frac{b(i) - a(i)}{\max\{a(i), b(i)\}}
    \end{equation}
    donde $a(i)$ es la distancia promedio intra-cluster y $b(i)$ la distancia al cluster más cercano.
    
    \item \textbf{Índice de Davies-Bouldin}: Mide la similitud promedio entre clusters (menor es mejor).
    
    \item \textbf{Gap Statistic}: Compara la inercia observada con la esperada bajo datos aleatorios.
\end{enumerate}

\subsection{Inferencia Estadística}

\subsubsection{Intervalos de Confianza}

\begin{definicion}[Intervalo de Confianza para la Media]
Para una muestra de tamaño $n$ con media $\bar{x}$ y desviación estándar $s$, el IC del $(1-\alpha)100\%$ es:
\begin{equation}
    \bar{x} \pm t_{\alpha/2, n-1} \cdot \frac{s}{\sqrt{n}}
\end{equation}
donde $t_{\alpha/2, n-1}$ es el cuantil de la distribución t de Student.
\end{definicion}

\subsubsection{Prueba t de Student}

\begin{definicion}[Test t para una muestra]
Para probar $H_0: \mu = \mu_0$ contra $H_a: \mu \neq \mu_0$:
\begin{equation}
    t = \frac{\bar{x} - \mu_0}{s / \sqrt{n}}
\end{equation}
El estadístico $t$ sigue una distribución t de Student con $n-1$ grados de libertad.
\end{definicion}

\begin{definicion}[Test t de Welch]
Para comparar dos muestras independientes con varianzas potencialmente diferentes:
\begin{equation}
    t = \frac{\bar{x}_1 - \bar{x}_2}{\sqrt{\frac{s_1^2}{n_1} + \frac{s_2^2}{n_2}}}
\end{equation}
\end{definicion}

\subsubsection{Bootstrap}

\begin{definicion}[Bootstrap Percentil]
Método no paramétrico para estimar la distribución muestral de un estadístico:
\begin{enumerate}
    \item Extraer $B$ muestras con reemplazo de tamaño $n$ de los datos originales.
    \item Calcular el estadístico de interés para cada muestra.
    \item Usar los percentiles $\alpha/2$ y $1-\alpha/2$ de la distribución empírica.
\end{enumerate}
\end{definicion}

% ============================================
% CAPÍTULO 3: DATOS
% ============================================
\section{Datos y Recopilación}

\subsection{Fuente de Datos}

Los datos fueron extraídos de \textbf{Yahoo Finance} mediante la librería \texttt{yfinance} de Python. Esta fuente proporciona:
\begin{itemize}
    \item Precios históricos ajustados por splits y dividendos
    \item Volumen de transacciones
    \item Datos desde 2018 hasta 2024
\end{itemize}

\subsection{Empresas Analizadas}

Se seleccionaron 30 empresas tecnológicas líderes a nivel mundial:

\begin{table}[H]
\centering
\small
\begin{tabular}{llll}
\toprule
\textbf{Ticker} & \textbf{Empresa} & \textbf{Ticker} & \textbf{Empresa} \\
\midrule
AAPL & Apple & NVDA & Nvidia \\
MSFT & Microsoft & TSLA & Tesla \\
GOOGL & Alphabet (Google) & META & Meta (Facebook) \\
AMZN & Amazon & NFLX & Netflix \\
TSM & Taiwan Semiconductor & ASML & ASML \\
AVGO & Broadcom & CRM & Salesforce \\
ADBE & Adobe & ORCL & Oracle \\
CSCO & Cisco & INTC & Intel \\
IBM & IBM & ACN & Accenture \\
NOW & ServiceNow & SAP & SAP \\
INFY & Infosys & SONY & Sony \\
PLTR & Palantir & SNOW & Snowflake \\
SPOT & Spotify & NET & Cloudflare \\
FTNT & Fortinet & WDAY & Workday \\
005930.KS & Samsung & 0700.HK & Tencent \\
\bottomrule
\end{tabular}
\caption{Lista de empresas tecnológicas analizadas}
\end{table}

\subsection{Variables del Dataset}

\begin{table}[H]
\centering
\begin{tabular}{lp{8cm}l}
\toprule
\textbf{Variable} & \textbf{Descripción} & \textbf{Tipo} \\
\midrule
\texttt{Company} & Nombre de la empresa & Categórica \\
\texttt{Ticker} & Símbolo bursátil & Categórica \\
\texttt{Date} & Fecha (fin de mes) & Temporal \\
\texttt{AdjClose} & Precio ajustado de cierre & Numérica \\
\texttt{Volume} & Volumen mensual negociado & Numérica \\
\texttt{Return} & Rendimiento logarítmico mensual & Numérica \\
\texttt{MeanReturn} & Rendimiento promedio (agregado) & Numérica \\
\texttt{Volatility} & Desviación estándar de rendimientos & Numérica \\
\texttt{Beta} & Coeficiente beta CAPM & Numérica \\
\bottomrule
\end{tabular}
\caption{Descripción de las variables del dataset}
\end{table}

\subsection{Período de Análisis}

\begin{itemize}
    \item \textbf{Período}: Enero 2018 - Diciembre 2024 (83 meses)
    \item \textbf{Frecuencia}: Mensual
    \item \textbf{Observaciones}: Aproximadamente 2,490 observaciones (30 empresas × 83 meses)
    \item \textbf{Índice de referencia}: QQQ (ETF del NASDAQ-100)
\end{itemize}

% ============================================
% CAPÍTULO 4: EDA
% ============================================
\section{Análisis Exploratorio de Datos (EDA)}

El análisis exploratorio tiene como objetivo comprender la estructura de los datos antes de aplicar técnicas estadísticas avanzadas.

\subsection{Estadísticos Descriptivos}

\subsubsection{Distribución de Rendimientos}

Los rendimientos logarítmicos mensuales presentan las siguientes características:

\begin{table}[H]
\centering
\begin{tabular}{lrrrrrr}
\toprule
\textbf{Estadístico} & \textbf{AdjClose} & \textbf{Volume} & \textbf{Return} \\
\midrule
Media & Variable & Variable & 0.015 \\
Desv. Est. & Variable & Variable & 0.105 \\
Mínimo & Variable & Variable & -0.50 \\
Máximo & Variable & Variable & 0.60 \\
\bottomrule
\end{tabular}
\caption{Estadísticos descriptivos de las variables principales}
\end{table}

\subsubsection{Ranking de Empresas por Rendimiento}

Las empresas con mayor rendimiento promedio mensual fueron:
\begin{enumerate}
    \item \textbf{Palantir (PLTR)}: 4.11\% mensual
    \item \textbf{Nvidia (NVDA)}: 3.76\% mensual
    \item \textbf{Tesla (TSLA)}: 3.46\% mensual
    \item \textbf{Broadcom (AVGO)}: 2.98\% mensual
    \item \textbf{Cloudflare (NET)}: 2.81\% mensual
\end{enumerate}

\subsection{Visualizaciones Clave}

\subsubsection{Distribución Global de Rendimientos}

\begin{figure}[H]
\centering
\begin{tikzpicture}
\begin{axis}[
    ybar interval,
    xlabel={Rendimiento mensual},
    ylabel={Frecuencia},
    title={Histograma de rendimientos mensuales (todas las empresas)},
    width=12cm,
    height=7cm,
    grid=major,
    xmin=-0.35,
    xmax=0.30
]
\addplot+[hist={bins=10, data min=-0.3, data max=0.25}] coordinates {
    (-0.30, 0) (-0.25, 0) (-0.20, 0) (-0.15, 0) (-0.10, 0)
    (-0.05, 0) (0.00, 0) (0.05, 0) (0.10, 0) (0.15, 0)
    (0.20, 0) (0.25, 0)
};
\end{axis}
\end{tikzpicture}
\caption{Distribución de rendimientos mensuales (aproximación)}
\end{figure}

\subsubsection{Relación Riesgo-Rendimiento}

Una de las relaciones fundamentales en finanzas es la relación entre riesgo (volatilidad) y rendimiento esperado:

\begin{figure}[H]
\centering
\begin{tikzpicture}
\begin{axis}[
    xlabel={Volatilidad (Desv. Est. mensual)},
    ylabel={Rendimiento medio mensual},
    title={Frontera Riesgo-Rendimiento de las 30 empresas},
    width=12cm,
    height=8cm,
    grid=major,
    scatter,
    only marks,
    scatter src=explicit symbolic
]
\addplot[scatter, mark=*, blue!70] coordinates {
    (0.059, 0.019) % MSFT
    (0.084, 0.022) % AAPL
    (0.073, 0.014) % GOOGL
    (0.089, 0.013) % AMZN
    (0.114, 0.014) % META
    (0.140, 0.038) % NVDA
    (0.189, 0.035) % TSLA
    (0.097, 0.016) % ASML
    (0.092, 0.030) % AVGO
    (0.071, 0.007) % CSCO
    (0.184, 0.028) % NET
    (0.248, 0.041) % PLTR
    (0.125, 0.014) % NFLX
};
\end{axis}
\end{tikzpicture}
\caption{Scatter plot de volatilidad vs. rendimiento medio}
\end{figure}

\subsubsection{Boxplots por Empresa}

Los boxplots revelan la dispersión de rendimientos y la presencia de valores atípicos (outliers) para cada empresa.

\subsection{Correlaciones}

La matriz de correlación entre las variables numéricas principales muestra:
\begin{itemize}
    \item Correlación moderada entre rendimiento y volumen
    \item Las acciones tecnológicas están altamente correlacionadas entre sí
    \item El índice QQQ captura bien el movimiento general del sector
\end{itemize}

\subsection{Evaluación de Normalidad}

Se aplicaron tests de normalidad (Shapiro-Wilk) a los rendimientos de cada empresa:

\begin{importante}[Resultado del Test de Normalidad]
La mayoría de las empresas muestran desviaciones significativas de la normalidad (p < 0.05), justificando el uso de:
\begin{itemize}
    \item Pruebas no paramétricas (Mann-Whitney)
    \item Métodos de bootstrap
    \item Errores estándar robustos en regresión
\end{itemize}
\end{importante}

% ============================================
% CAPÍTULO 5: REGRESIÓN
% ============================================
\section{Análisis de Regresión: Modelo CAPM}

\subsection{Objetivo}

Estimar el coeficiente beta ($\beta$) para cada empresa, que mide la sensibilidad de los rendimientos de la acción respecto a los rendimientos del mercado (QQQ).

\subsection{Metodología}

Para cada empresa $i$, se estima el modelo:
\begin{equation}
    R_{i,t} = \alpha_i + \beta_i \cdot R_{m,t} + \gamma_i \cdot \log(\text{Volume}_{i,t}) + \epsilon_{i,t}
\end{equation}

Donde:
\begin{itemize}
    \item $R_{i,t}$: Rendimiento de la empresa $i$ en el mes $t$
    \item $R_{m,t}$: Rendimiento del mercado (QQQ) en el mes $t$
    \item $\log(\text{Volume}_{i,t})$: Logaritmo del volumen negociado
\end{itemize}

\subsection{Implementación}

\begin{lstlisting}[language=Python, caption={Código de estimación CAPM}]
import statsmodels.api as sm

def estimate_capm(company, panel_df, market_returns):
    df_i = panel_df[panel_df["Company"] == company]
    df = df_i.merge(market_returns, on="Date")
    
    X = sm.add_constant(df[["MarketReturn", "logVol"]])
    y = df["Return"]
    
    model = sm.OLS(y, X).fit(cov_type="HC1")
    return model.params["MarketReturn"]  # beta
\end{lstlisting}

\subsection{Resultados}

\begin{table}[H]
\centering
\begin{tabular}{lccc}
\toprule
\textbf{Empresa} & \textbf{Beta ($\beta$)} & \textbf{Interpretación} & \textbf{R²} \\
\midrule
Palantir (PLTR) & 2.14 & Muy agresivo & 0.45 \\
Tesla (TSLA) & 1.89 & Muy agresivo & 0.52 \\
Nvidia (NVDA) & 1.88 & Muy agresivo & 0.60 \\
Spotify (SPOT) & 1.60 & Agresivo & 0.48 \\
Netflix (NFLX) & 1.39 & Agresivo & 0.55 \\
\midrule
Microsoft (MSFT) & 0.86 & Defensivo & 0.72 \\
Cisco (CSCO) & 0.68 & Defensivo & 0.45 \\
IBM (IBM) & 0.59 & Defensivo & 0.38 \\
Tencent (0700.HK) & 0.40 & Muy defensivo & 0.15 \\
\bottomrule
\end{tabular}
\caption{Resultados del modelo CAPM por empresa (selección)}
\end{table}

\subsection{Diagnósticos del Modelo}

Para validar los supuestos de la regresión lineal:

\begin{enumerate}
    \item \textbf{Durbin-Watson}: Evalúa autocorrelación residual
    \begin{itemize}
        \item Valores cercanos a 2 indican ausencia de autocorrelación
        \item Promedio observado: $\approx 1.9$
    \end{itemize}
    
    \item \textbf{Test de Breusch-Pagan}: Evalúa heterocedasticidad
    \begin{itemize}
        \item Si $p < 0.05$: presencia de heterocedasticidad
        \item Solución: uso de errores estándar robustos (HC1)
    \end{itemize}
\end{enumerate}

\subsection{Interpretación de Resultados}

\begin{concepto}[Interpretación del Beta]
\begin{itemize}
    \item \textbf{Nvidia ($\beta = 1.88$)}: Por cada 1\% que sube el mercado, Nvidia sube en promedio 1.88\%.
    \item \textbf{Microsoft ($\beta = 0.86$)}: Por cada 1\% que sube el mercado, Microsoft sube solo 0.86\%.
    \item Las empresas con $\beta > 1$ amplifican los movimientos del mercado (mayor riesgo y potencial retorno).
\end{itemize}
\end{concepto}

% ============================================
% CAPÍTULO 6: CLUSTERING
% ============================================
\section{Análisis de Clustering}

\subsection{Objetivo}

Identificar grupos naturales de empresas con perfiles de riesgo-rendimiento similares, sin utilizar etiquetas predefinidas.

\subsection{Variables Utilizadas}

Para el clustering se utilizaron las siguientes variables (estandarizadas):
\begin{enumerate}
    \item \textbf{MeanReturn}: Rendimiento promedio mensual
    \item \textbf{Volatility}: Desviación estándar de rendimientos
    \item \textbf{Beta}: Sensibilidad al mercado
    \item \textbf{AvgVolume}: Volumen promedio negociado
\end{enumerate}

\subsection{Preprocesamiento}

\begin{lstlisting}[language=Python, caption={Estandarización de variables}]
from sklearn.preprocessing import StandardScaler

features = ['MeanReturn', 'Volatility', 'Beta', 'AvgVolume']
scaler = StandardScaler()
X_scaled = scaler.fit_transform(agg_df[features])
\end{lstlisting}

La estandarización es crucial para que todas las variables contribuyan equitativamente a las distancias euclídeas.

\subsection{Selección del Número Óptimo de Clusters}

Se evaluaron tres métricas complementarias:

\begin{figure}[H]
\centering
\begin{tikzpicture}
\begin{axis}[
    xlabel={Número de Clusters (k)},
    ylabel={Silhouette Score},
    title={Método del Silhouette para selección de k},
    width=10cm,
    height=6cm,
    grid=major,
    xtick={2,3,4,5,6,7,8,9,10}
]
\addplot[thick, blue, mark=*] coordinates {
    (2, 0.35)
    (3, 0.42)
    (4, 0.38)
    (5, 0.32)
    (6, 0.28)
    (7, 0.25)
    (8, 0.22)
    (9, 0.20)
    (10, 0.18)
};
\end{axis}
\end{tikzpicture}
\caption{Silhouette Score para diferentes valores de k}
\end{figure}

\begin{table}[H]
\centering
\begin{tabular}{lc}
\toprule
\textbf{Métrica} & \textbf{k óptimo} \\
\midrule
Gap Statistic & 3 \\
Silhouette Score & 3 \\
Davies-Bouldin Index & 3 \\
\bottomrule
\end{tabular}
\caption{Número óptimo de clusters según diferentes métricas}
\end{table}

\subsection{Resultados del Clustering}

Con $k = 3$, se identificaron los siguientes perfiles:

\begin{table}[H]
\centering
\begin{tabular}{cp{10cm}}
\toprule
\textbf{Cluster} & \textbf{Características} \\
\midrule
\textbf{Cluster 0} & \textbf{Defensivas estables}: Empresas con bajo beta (<1), baja volatilidad y rendimientos moderados. Ejemplos: Microsoft, Cisco, IBM, Accenture. \\
\midrule
\textbf{Cluster 1} & \textbf{Crecimiento moderado}: Empresas con beta cercano a 1, volatilidad media. Ejemplos: Apple, Google, Amazon, Adobe. \\
\midrule
\textbf{Cluster 2} & \textbf{Alto crecimiento/riesgo}: Empresas con alto beta (>1.5), alta volatilidad pero también altos rendimientos. Ejemplos: Tesla, Nvidia, Palantir. \\
\bottomrule
\end{tabular}
\caption{Caracterización de los clusters identificados}
\end{table}

\subsection{Visualización en 2D (PCA)}

Para visualizar los clusters en 2D, se aplicó Análisis de Componentes Principales:

\begin{figure}[H]
\centering
\begin{tikzpicture}
\begin{axis}[
    xlabel={Componente Principal 1},
    ylabel={Componente Principal 2},
    title={Clusters en espacio PCA},
    width=12cm,
    height=8cm,
    grid=major,
    legend pos=outer north east
]

% Cluster 0 - Defensivas
\addplot[only marks, mark=*, blue!70, mark size=3pt] coordinates {
    (-1.5, -0.5) (-1.2, -0.3) (-1.0, 0.2) (-0.8, -0.4) (-1.3, 0.1)
};
\addlegendentry{Defensivas}

% Cluster 1 - Moderadas
\addplot[only marks, mark=square*, green!60!black, mark size=3pt] coordinates {
    (-0.2, 0.3) (0.1, -0.2) (0.3, 0.5) (-0.1, -0.4) (0.4, 0.1) (0.2, -0.1)
};
\addlegendentry{Crecimiento moderado}

% Cluster 2 - Alto riesgo
\addplot[only marks, mark=triangle*, red!70, mark size=4pt] coordinates {
    (1.5, 1.2) (2.0, 0.8) (1.8, 1.5) (2.2, 0.5) (1.3, 0.9)
};
\addlegendentry{Alto crecimiento}

\end{axis}
\end{tikzpicture}
\caption{Visualización de clusters mediante PCA}
\end{figure}

% ============================================
% CAPÍTULO 7: INFERENCIA
% ============================================
\section{Inferencia Estadística}

\subsection{Objetivos de la Inferencia}

\begin{enumerate}
    \item Construir intervalos de confianza para los rendimientos medios
    \item Probar si los rendimientos son significativamente diferentes de cero
    \item Comparar grupos de empresas (alto beta vs. bajo beta)
    \item Aplicar bootstrap para robustez
\end{enumerate}

\subsection{Intervalos de Confianza para la Media}

\subsubsection{Método Clásico (t de Student)}

Para cada empresa, se calculó el IC del 95\% para el rendimiento medio:
\begin{equation}
    IC_{95\%} = \bar{r} \pm t_{0.025, n-1} \cdot \frac{s}{\sqrt{n}}
\end{equation}

\subsubsection{Método Bootstrap}

Debido a las desviaciones de normalidad, también se calcularon ICs mediante bootstrap percentil con $B = 5000$ remuestreos.

\begin{table}[H]
\centering
\small
\begin{tabular}{lcccc}
\toprule
\textbf{Empresa} & \textbf{Media} & \textbf{IC t (95\%)} & \textbf{IC Bootstrap (95\%)} \\
\midrule
Nvidia & 0.0376 & [0.0071, 0.0681] & [0.0085, 0.0672] \\
Tesla & 0.0346 & [-0.0065, 0.0757] & [-0.0048, 0.0743] \\
Microsoft & 0.0190 & [0.0061, 0.0319] & [0.0068, 0.0312] \\
Apple & 0.0224 & [0.0041, 0.0407] & [0.0052, 0.0398] \\
\bottomrule
\end{tabular}
\caption{Intervalos de confianza comparativos}
\end{table}

\subsection{Prueba t de Una Muestra}

\begin{concepto}[Hipótesis]
\begin{align}
    H_0&: \mu = 0 \text{ (rendimiento medio es cero)} \\
    H_a&: \mu \neq 0 \text{ (rendimiento medio es diferente de cero)}
\end{align}
\end{concepto}

\begin{table}[H]
\centering
\begin{tabular}{lccc}
\toprule
\textbf{Empresa} & \textbf{t-stat} & \textbf{p-value} & \textbf{Decisión ($\alpha$=0.05)} \\
\midrule
Nvidia & 2.45 & 0.016 & Rechazar $H_0$ \\
Microsoft & 2.92 & 0.004 & Rechazar $H_0$ \\
Apple & 2.44 & 0.017 & Rechazar $H_0$ \\
Intel & -0.75 & 0.456 & No rechazar $H_0$ \\
Samsung & 0.37 & 0.712 & No rechazar $H_0$ \\
\bottomrule
\end{tabular}
\caption{Resultados del test t de una muestra}
\end{table}

\subsection{Comparación de Grupos: Alto Beta vs. Bajo Beta}

Se dividieron las empresas en dos grupos según su coeficiente beta:
\begin{itemize}
    \item \textbf{Grupo Alto Beta}: $\beta > 1.0$ (empresas agresivas)
    \item \textbf{Grupo Bajo Beta}: $\beta \leq 1.0$ (empresas defensivas)
\end{itemize}

\subsubsection{Test t de Welch}

\begin{align}
    H_0&: \mu_{\text{alto}} = \mu_{\text{bajo}} \\
    H_a&: \mu_{\text{alto}} \neq \mu_{\text{bajo}}
\end{align}

\begin{lstlisting}[language=Python, caption={Implementación del test de Welch}]
from scipy import stats

high_beta = agg_df[agg_df['Beta'] > 1.0]['MeanReturn']
low_beta = agg_df[agg_df['Beta'] <= 1.0]['MeanReturn']

t_stat, p_value = stats.ttest_ind(high_beta, low_beta, equal_var=False)
\end{lstlisting}

\subsubsection{Test de Mann-Whitney (No Paramétrico)}

Como alternativa robusta cuando no se cumple normalidad:
\begin{equation}
    U = n_1 n_2 + \frac{n_1(n_1+1)}{2} - R_1
\end{equation}

\subsection{Bootstrap para Diferencia de Medias}

Se implementó un test de permutación bootstrap para comparar los rendimientos entre grupos:

\begin{lstlisting}[language=Python, caption={Bootstrap para diferencia de medias}]
def bootstrap_diff_means(x1, x2, n_boot=5000):
    obs_diff = x1.mean() - x2.mean()
    pooled = np.concatenate([x1, x2])
    
    boot_diffs = []
    for _ in range(n_boot):
        perm = np.random.permutation(pooled)
        boot_diffs.append(perm[:len(x1)].mean() - perm[len(x1):].mean())
    
    p_value = np.mean(np.abs(boot_diffs) >= np.abs(obs_diff))
    return obs_diff, p_value
\end{lstlisting}

\subsection{Corrección por Comparaciones Múltiples}

Al realizar múltiples pruebas de hipótesis, aumenta el riesgo de errores Tipo I. Se aplicaron correcciones:

\begin{itemize}
    \item \textbf{Bonferroni}: $\alpha_{\text{ajustado}} = \alpha / m$
    \item \textbf{Benjamini-Hochberg (FDR)}: Controla la tasa de falsos descubrimientos
\end{itemize}

% ============================================
% CAPÍTULO 8: RESULTADOS Y CONCLUSIONES
% ============================================
\section{Resultados y Conclusiones}

\subsection{Síntesis de Hallazgos}

\begin{importante}[Respuestas a las Preguntas de Investigación]
\textbf{P1: ¿Cómo se relaciona el rendimiento con el mercado?}
\begin{itemize}
    \item El modelo CAPM explica entre 15\% y 72\% de la variación en rendimientos según la empresa.
    \item Los betas oscilan entre 0.40 (Tencent) y 2.14 (Palantir).
    \item Las empresas más nuevas/volátiles (Tesla, Nvidia, Palantir) tienen betas > 1.5.
\end{itemize}

\textbf{P2: ¿Existen grupos naturales de empresas?}
\begin{itemize}
    \item Sí, se identificaron 3 clusters: defensivas, crecimiento moderado, alto crecimiento.
    \item El silhouette score de 0.42 indica una separación moderada pero significativa.
\end{itemize}

\textbf{P3: ¿Hay diferencias significativas entre grupos de riesgo?}
\begin{itemize}
    \item El test de Welch muestra diferencias significativas (p < 0.05) entre alto y bajo beta.
    \item Las empresas de alto beta tienen rendimientos medios más altos pero mayor variabilidad.
\end{itemize}
\end{importante}

\subsection{Limitaciones del Estudio}

\begin{enumerate}
    \item \textbf{Período temporal}: El período 2018-2024 incluye eventos extraordinarios (COVID-19, crisis de 2022).
    
    \item \textbf{Supuestos del CAPM}: 
    \begin{itemize}
        \item No considera tasa libre de riesgo variable
        \item Asume relación lineal entre rendimientos
    \end{itemize}
    
    \item \textbf{Tamaño muestral}: 30 empresas es un tamaño moderado para clustering robusto.
    
    \item \textbf{Sector homogéneo}: Todas son tecnológicas, lo que limita la generalización.
\end{enumerate}

\subsection{Trabajo Futuro}

\begin{enumerate}
    \item Extender el análisis a otros sectores (financiero, salud, energía)
    \item Implementar modelos multifactoriales (Fama-French de 3 o 5 factores)
    \item Aplicar modelos de volatilidad condicional (GARCH)
    \item Incluir análisis de series temporales (estacionariedad, cointegración)
\end{enumerate}

\subsection{Conclusión General}

Este proyecto demuestra la aplicación efectiva de técnicas estadísticas fundamentales al análisis financiero. Los resultados confirman que:

\begin{enumerate}
    \item El modelo CAPM, aunque simplificado, captura relaciones significativas entre acciones individuales y el mercado.
    \item Las técnicas de clustering revelan estructuras naturales en los datos financieros.
    \item La inferencia estadística robusta (incluyendo bootstrap) es esencial dado que los rendimientos financieros raramente siguen distribuciones normales exactas.
\end{enumerate}

% ============================================
% DIAGRAMA DE FLUJO DEL PROYECTO
% ============================================
\newpage
\section*{Apéndice A: Flujo del Proyecto}
\addcontentsline{toc}{section}{Apéndice A: Flujo del Proyecto}

\begin{figure}[H]
\centering
\begin{tikzpicture}[
    node distance=1.5cm,
    startstop/.style={rectangle, rounded corners, minimum width=3cm, minimum height=1cm, text centered, draw=black, fill=red!30},
    process/.style={rectangle, minimum width=3cm, minimum height=1cm, text centered, draw=black, fill=blue!20},
    decision/.style={diamond, minimum width=3cm, minimum height=1cm, text centered, draw=black, fill=green!20},
    arrow/.style={thick, ->, >=stealth}
]

% Nodos
\node (start) [startstop] {Inicio: Definir preguntas};
\node (data) [process, below of=start] {Extracción de datos (yfinance)};
\node (eda) [process, below of=data] {Análisis Exploratorio (EDA)};
\node (prep) [process, below of=eda] {Preparación de datos};
\node (reg) [process, below of=prep] {Regresión CAPM};
\node (cluster) [process, below of=reg] {Clustering K-Means};
\node (inf) [process, below of=cluster] {Inferencia Estadística};
\node (end) [startstop, below of=inf] {Conclusiones};

% Flechas
\draw [arrow] (start) -- (data);
\draw [arrow] (data) -- (eda);
\draw [arrow] (eda) -- (prep);
\draw [arrow] (prep) -- (reg);
\draw [arrow] (reg) -- (cluster);
\draw [arrow] (cluster) -- (inf);
\draw [arrow] (inf) -- (end);

\end{tikzpicture}
\caption{Diagrama de flujo del proyecto}
\end{figure}

% ============================================
% APÉNDICE B: FÓRMULAS CLAVE
% ============================================
\section*{Apéndice B: Fórmulas Estadísticas Clave}
\addcontentsline{toc}{section}{Apéndice B: Fórmulas Estadísticas Clave}

\begin{table}[H]
\centering
\begin{tabular}{p{4cm}p{8cm}}
\toprule
\textbf{Concepto} & \textbf{Fórmula} \\
\midrule
Rendimiento logarítmico & $r_t = \ln(P_t / P_{t-1})$ \\[0.3cm]
Volatilidad & $\sigma = \sqrt{\frac{1}{n-1}\sum(r_t - \bar{r})^2}$ \\[0.3cm]
Coeficiente Beta & $\beta = \frac{\text{Cov}(R_i, R_m)}{\text{Var}(R_m)}$ \\[0.3cm]
IC para la media (t) & $\bar{x} \pm t_{\alpha/2, n-1} \cdot \frac{s}{\sqrt{n}}$ \\[0.3cm]
Estadístico t & $t = \frac{\bar{x} - \mu_0}{s/\sqrt{n}}$ \\[0.3cm]
Silhouette Score & $s(i) = \frac{b(i) - a(i)}{\max\{a(i), b(i)\}}$ \\[0.3cm]
\bottomrule
\end{tabular}
\caption{Resumen de fórmulas estadísticas utilizadas}
\end{table}

% ============================================
% REFERENCIAS
% ============================================
\newpage
\section*{Referencias Bibliográficas}
\addcontentsline{toc}{section}{Referencias Bibliográficas}

\begin{thebibliography}{99}

\bibitem{sharpe1964}
Sharpe, W. F. (1964). Capital Asset Prices: A Theory of Market Equilibrium under Conditions of Risk. \textit{The Journal of Finance}, 19(3), 425-442.

\bibitem{markowitz1952}
Markowitz, H. (1952). Portfolio Selection. \textit{The Journal of Finance}, 7(1), 77-91.

\bibitem{campbell1997}
Campbell, J. Y., Lo, A. W., \& MacKinlay, A. C. (1997). \textit{The Econometrics of Financial Markets}. Princeton University Press.

\bibitem{fama1993}
Fama, E. F., \& French, K. R. (1993). Common risk factors in the returns on stocks and bonds. \textit{Journal of Financial Economics}, 33(1), 3-56.

\bibitem{efron1993}
Efron, B., \& Tibshirani, R. J. (1993). \textit{An Introduction to the Bootstrap}. Chapman \& Hall/CRC.

\bibitem{james2013}
James, G., Witten, D., Hastie, T., \& Tibshirani, R. (2013). \textit{An Introduction to Statistical Learning}. Springer.

\bibitem{hastie2009}
Hastie, T., Tibshirani, R., \& Friedman, J. (2009). \textit{The Elements of Statistical Learning} (2nd ed.). Springer.

\bibitem{wooldridge2016}
Wooldridge, J. M. (2016). \textit{Introductory Econometrics: A Modern Approach} (6th ed.). Cengage Learning.

\bibitem{statsmodels}
Seabold, S., \& Perktold, J. (2010). Statsmodels: Econometric and Statistical Modeling with Python. \textit{Proceedings of the 9th Python in Science Conference}.

\bibitem{sklearn}
Pedregosa, F., et al. (2011). Scikit-learn: Machine Learning in Python. \textit{Journal of Machine Learning Research}, 12, 2825-2830.

\bibitem{yfinance}
Aroussi, R. (2023). yfinance: Download market data from Yahoo! Finance API. \url{https://github.com/ranaroussi/yfinance}

\end{thebibliography}

% ============================================
% FIN DEL DOCUMENTO
% ============================================
\end{document}
